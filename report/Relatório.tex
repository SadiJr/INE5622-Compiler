\documentclass[12pt,a4paper,brazil,abntex2]{article}
\usepackage{lmodern}			% Usa a fonte Latin Modern			
\usepackage[T1]{fontenc}		% Selecao de codigos de fonte.
\usepackage[utf8]{inputenc}		% Codificacao do documento (conversão automática dos acentos)
\usepackage[brazil]{babel}
\usepackage{indentfirst}		% Indenta o primeiro parágrafo de cada seção.
\usepackage{graphicx}			% Inclusão de gráficos
\usepackage{microtype} 			% para melhorias de justificação
\usepackage[left=3cm,right=2cm,top=3cm,bottom=2cm]{geometry}
\usepackage{url}
\usepackage[hidelinks]{hyperref}
\usepackage{setspace}
\usepackage{cite}

\begin{document}
\singlespacing
\begin{titlepage}
\begin{center}
\begin{figure}[!htb]
\center

\includegraphics[scale=0.25]{~/Curso/Brasao/Sigla.pdf} 

\end{figure}
{\bf  UNIVERSIDADE FEDERAL DE SANTA CATARINA}\\[0.2cm] %0,2cm é a distância entre o texto dessa linha e o texto da próxima
{\bf CENTRO TECNOLÓGICO}\\[0.2cm] % o comando \\ "manda" o texto ir para próxima linha
{\bf  DEPARTAMENTO DE INFORMÁTICA E ESTATÍSTICA}\\[5.5cm]
{\bf \large IMPLEMENTAÇÃO DE COMPILADOR}\\[3.8cm] % o comando \bf deixa o texto entre chaves em negrito. O comando \huge deixa o texto enorme
{Marcos Silva Laydner}\\
{Nathan Sargon Werlich}\\
{Higor Nocetti}\\
{Sadi Júnior Domingos Jacinto}\\[0.7cm] % o comando \large deixa o texto grande
{Professor orientador: Rafael de Santiago}\\[4cm]
{Florianópolis}\\[0.2cm]
{2020}
\newpage
\thispagestyle{empty}
{Marcos Silva Laydner}\\
{Nathan Sargon Werlich}\\
{Higor Nocetti}\\
{Sadi Júnior Domingos Jacinto}\\[9cm] % o comando \large deixa o texto grande
{\bf \large IMPLEMENTAÇÃO DE COMPILADOR}\\[0.5cm]
    \begin{flushright}
    \begin{list}{}{
      \setlength{\leftmargin}{7.2cm}
      \setlength{\rightmargin}{0cm}
      \setlength{\labelwidth}{0pt}
      \setlength{\labelsep}{\leftmargin}}
      \item Trabalho prático da disciplina INE5622 – Introdução a Compiladores, consistindo na implementação de um compilador (analisador léxico e sintático), com o uso da ferramenta ANTLR4, necessário para obtenção de nota.\\[0.2 cm] 
      \setlength{\labelsep}{\leftmargin}
      \item Professor orientador: Rafael de Santiago\
      \\[8.2cm]
     \end{list}
	 \end{flushright}
{Florianópolis}\\[0.2cm]
{2020}
\end{center}
\end{titlepage} %término da "capa"
\newpage
\thispagestyle{empty}
\begin{center}
\tableofcontents
\end{center}

\section{\normalsize CONTRIBUIÇÃO DOS MEMBROS}
	\begin{table}[h]
		\begin{center}
		\begin{tabular}{|c|c|}
		\hline
			{\bf Participante} \	& {\bf Contribuição}\\\hline
			Marcos Silva Laydner & \\\hline
			Nathan Sargon Werlich & \\\hline
			Higor Nocetti & \\\hline
			Sadi Júnior Domingos Jacinto & Elaboração do Relatório\\\hline
		\end{tabular}
		\end{center}
	\end{table}
\section{\normalsize DESCRIÇÃO DA LINGUAGEM}
\section{\normalsize EXEMPLOS DE CÓDIGOS}
\end{document}