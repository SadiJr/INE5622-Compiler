\section{\normalsize IMPLEMENTAÇÃO DO COMPILADOR}
	A implementação do compilador se baseou principalmente na linguagem de exemplo da disciplina, sendo o resto da implementação baseada em pura força-bruta\footnote{A implementação de todas as características não contempladas na linguagem de exemplo da disciplina foi por tentativa e erro.}.
	
	\subsection{\normalsize CARACTERÍSTICAS FUNCIONAIS DO COMPILADOR}
	\begin{itemize}
		\item Tipos \textit{int} e \textit{float} implementados corretamente;
		\item Possibilidade de definir e chamar funções;
		\item Estruturas \textit{if-then-else}, \textit{while}, \textit{for} e \textit{switch-case} funcionais;
		\item Operações matemáticas e lógicas funcionais, para ambos os tipos (\textit{int} e \textit{float});
		\item Toda a interface de comando (CLI) exigida.
		
	\end{itemize}
	
	\subsection{\normalsize CARACTERÍSTICAS NÃO-FUNCIONAIS}
	\begin{itemize}
		\item O reporte de erros semânticos foi implementado, e testado, apenas parcialmente, podendo haver erros ainda não detectados e/ou tratados.
		\item Fora os tipos \textit{int} e \textit{float}, nenhum outro tipo foi implementado. Além disso, somente as operações matemáticas básicas (adição, subtração, divisão e multiplicação) foram implementadas, mesmo que a definição da linguagem suporte a operação de módulo.
	\end{itemize}
	
	\subsection{\normalsize LIMITAÇÕES E ERROS DO COMPILADOR}
	\begin{itemize}
		\item Alguns erros sintáticos podem ser falsos positivos, porém, esses erros não interrompem o processo de compilação, apenas emitem uma mensagem de erro.
		\item Não é possível executar operações ou declarar variáveis fora de funções;
		\item Não é possível chamar funções dentro de funções\footnote{Exemplo: print(teste(teste2));}, nem passar para funções parâmetros que consistem em operações matemáticas ou lógicas;
		\item Obrigatoriamente deve haver uma função nomeada \textit{main};
		\item Não é possível ter duas variáveis com o mesmo nome em funções diferentes, desde que uma dessas funções seja a \textit{main};
		\item Todas as funções precisam ter \textit{return};
		\item Para uma função poder chamar a outra, a função a ser chamada precisa ser definida antes;
		\item Não é possível declarar uma variável de um tipo e depois a alterar para outro tipo;
		\item Operações de \textit{loop} usando variáveis do tipo \textit{float}, especialmente o \textit{while}, podem apresentar comportamentos estranhos, embora o resultado final apresentado não esteja totalmente errado.
	\end{itemize}
	
	\subsection{\normalsize NOTAS DO REDATOR}
	Todo o progresso de implementação do trabalho, assim como a contribuição e participação, em termos de código escrito e \textit{commits}, pode ser visualizado no seguinte repositório: \url{https://github.com/SadiJr/INE5622-Compiler}.

	Importante frisar que, a partir da postagem desse trabalho, tal repositório, antes privado, irá ter sua visibilidade alterada para público, o que pode, infelizmente, implicar em cópia por colegas desesperados.