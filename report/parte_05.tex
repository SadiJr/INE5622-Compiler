\section{\normalsize IMPLEMENTAÇÃO DO COMPILADOR}
	A implementação do compilador se baseou principalmente na linguagem de exemplo da disciplina, sendo o resto da implementação baseada em pura força-bruta.
	
	\subsection{\normalsize CARACTERÍSTICAS FUNCIONAIS DO COMPILADOR}
	\begin{itemize}
		\item Tipos \textit{int} e \textit{float} implementados corretamente;
		\item Possibilidade de definir e chamar funções;
		\item Estruturas \textit{if-then-else}, \textit{while}, \textit{for} e \textit{switch-case} funcionais;
		\item Operações matemáticas e lógicas funcionais, para ambos os tipos (\textit{int} e \textit{float});
		\item Toda a interface de comando (CLI) exigida.
		
	\end{itemize}
	
	\subsection{\normalsize CARACTERÍSTICAS NÃO-FUNCIONAIS}
	\begin{itemize}
		\item O reporte de erros semânticos foi implementado apenas parcialmente, sendo a maioria dos possíveis erros não tratados.;
		\item Fora os tipos \textit{int} e \textit{float}, nenhum outro tipo foi implementado. Além disso, somente as operações básicas (adição, subtração, divisão e multiplicação) foram implementadas, mesmo que a definição da linguagem suporte a operação de módulo.
	\end{itemize}
	
	\subsection{\normalsize LIMITAÇÕES E ERROS DO COMPILADOR}
	\begin{itemize}
		\item Não é possível executar operações ou declarar variáveis fora de funções;
		\item Não é possível chamar funções dentro de funções\footnote{Exemplo: print(teste(teste2));}, nem passar para funções parâmetros que consistem em operações matemáticas ou lógicas;
		\item Obrigatóriamente deve haver uma função nomeada \textit{main};
		\item Não é possível ter duas variáveis com o mesmo nome em funções diferentes, desde que uma dessas funções seja a \textit{main}.
		\item Todas as funções precisam ter \textit{return}.
		\item Para uma função poder chamar a outra, a função a ser chamada precisa ser definida antes.
		\item Não é possível passar parâmetros do tipo \textit{float} para as funções.
	\end{itemize}
	
	\subsection{\normalsize ERROR CONHECIDOS MAIS NÃO TRATADOS}
	\begin{itemize}
		\item Não é verificado se as funções possuem \textit{return}. Ao invés disso, um erro genérico, que em nada ajuda na detecção do erro, é lançado.
	\end{itemize}